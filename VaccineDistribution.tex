\documentclass{article}
\usepackage{gensymb}
\usepackage{graphicx}
\graphicspath{ {./res-images/} }
\usepackage{listings}
\begin{document}

\title{Building Resilience Vaccine Distribution}
\author{Saransh Sharma }
\date{January 2021}


\maketitle


\abstract

The COVID-19 pandemic has been one of the most influential pandemics in a century. Conventional life has been disrupted on a global scale, with rampant disease transmission leading to an unusual number of deaths. To take measures for this widespread virus transmission, vaccines have been developed since the early emergence of the pandemic. However, due to the substantial burden on the medical community and the gradual collapse of economies worldwide, vaccine testing processes were expedited to stabilize the uncontrolled virus. The development of some vaccines was accelerated by combining multiple phases of vaccine testing by running them in parallel. Once developed, the vaccines received early or limited approval. Yet, the distribution infrastructure and strategies remain undetermined, especially in necessary regions. It has impacted the timely delivery of vaccines to needy people and management to control the pandemic at an expected level. This study aims to address some of the deficiencies in the administration and distribution infrastructure of vaccines while proposing efficient and systematized strategies.

\section{Introduction}

The COVID-19 vaccines, Covaxin and AstraZeneca Oxford jab, in India, are approved for restricted use in an emergency situation in the public interest as an abundant precaution, in clinical trial mode, especially in the context of infection by mutant strains.\footnote{https://www.bbc.com/news/world-asia-india-55534902}, though there are serious concerns on lack of evidence and unsatisfactory scientific evidence of safety and efficacy for the vaccines produced in India.\footnote{bae 2020 challenges} 

India planned to vaccinate about 300 million people from January to July 2021, first vaccinating people above age 50. India is the second most populated nation, and most of the population is in rural and remote areas. The vaccination drive needs a strong vaccine intelligence network to ensure the last-mile delivery of the vaccine. Before that, there needs to be a dedicated focus on filling the gaps in the distribution network and resources. Presently, the world’s manufacturing processes, as well as supply chains, are significantly underprepared for the task of widespread vaccine distribution in a targeted short period of span [3]. 

COVID-19 vaccine movement in India possesses unknown and unresolved issues that have caused delays in the vaccination. Inefficiency in the supply chain and the lack of adequate infrastructure are primarily impacted the on-time delivery of vaccines at different locations. One of the major challenges in the process is a cold chain and distribution infrastructure. The existing cold chain in India works only for children and pregnant women vaccines that account for around 60 million people. As vaccine production is going on at a large scale, the existing infrastructure is incapable of storing vaccines required for 1.3 billion Indian people.  

Scaling up the cold chain distribution needs ten times more investments in procurement, distribution, allocation, functionality, and training. Moreover, some vaccines need to be stored at ultra-cold temperatures and should be used within a week. These vaccines can be stored for a longer period only in freezer storage. It is a big challenge to conduct such a large-scale delivery of vaccines with limited cold storage capacity and an inefficient distribution network. \footnote {https://www.bloombergquint.com/global-economics/india-faces-cold-chain-logistics- challenge-for-virus-vaccination}         

Moreover, it is essential to follow WHO guidelines for vaccine storage and distribution. According to WHO, the vaccine should be stored at 2-8\degree C [5] at every level of storage as the vaccine crosses different regions. The biological characteristic of the vaccine is impacted if it is exposed to a higher temperature than recommended. Such a vaccine may not be effective for disease prevention, ultimately increasing the demand for dosage and vaccine production. It is challenging in India to maintain and increase the capacity of a stable temperature i.e. cold chain for the existing vaccination program. Hence, coordinated efforts between regional levels are required to achieve a stable cold chain with an alternative approach that can offset the vaccine waste. 

Healthcare workers need to record users data at the site to optimize the management of vaccine distribution and supply chain. Studies have shown that several data capture challenges resulted in 10-60\% inaccurate data\cite{atkinson2020digital}. Lack of awareness regarding crucial data to be collected and access to easy tools in the existing workflow have been obstacles in data collection. Solutions and alternative interfaces to record data, such as vaccine products, administration routes, and vaccine product information, accurately are essential to enable data utilization across the information system for validation management. Data management across different level of local, regional and state. This data could be used across different verticals to make descions.

In this paper, we define different approaches and practical solutions to certain problems we have identified by reviwing the literature. We have defined these solutions in 

\begin{itemize}
	\item Synchronicity
	\item Event Layer
	\item AI For Finding Locations
	\item Cold Chain Alternative
	\item Use of Cryptographic Methods
\end{itemize}




\section{Synchronicity}
In order to capture and utilise data from different verticals, it is important to implement a synchronous data layer that updates data in real-time. Various services in these verticals create data attributes. A service that interacts with VIN sends a message to the receiving end that passes through a parser. The entire process can be stored as message streams or log for various auditable purposes in future. It is known as event sourcing, services recording all the states together as a sequence of events, such design allows to reconstruct past/ historical states. It is quite straightforward and less consuming than a querying database.

An indexer is implemented to identify the most frequently used attributes. These attributes can be stored as meta-data and can be used as a template when a service is sending out a message. For an eg: at a clinic, when a frontline worker records data about vaccines, he could use the metadata rather than a pre-defined field. The meta-data is useful to determine what kind of attributes are important from different verticals.

The general step of building a sync layer is to have a Parser. It cleans data once messages get through the parser validating data. The major goal of this layer is to reduce friction and establish communication between working groups and services.

Practical Applications such as RFID, sensors, Applications collecting data from delivery points, logistics and supply chain instruments that will create data at different vectors.

\includegraphics[scale=0.5]{hubb}


\begin{lstlisting}
	createData()
	parseData()
	processData()
		identify()
		extract()
		reshape()
	return()	
\end{lstlisting}

   
\section{Eventing Layer}

An eventing layer plays a vital role in the process that involves sending out important alerts and notices. The layer is used to identify and send notices and warnings to the interested parties and services involved in a process such as vaccines delivery to users and vaccines shipping from a location. All these processes have pre-defined steps, but in case of deviation in any of these steps, an eventing layer is supposed to send out notifications to involved parties.

Eventing layer service is used to manage end to end shipping and define event points that can be flow, or logical steps, or an incident. When an incident is triggered at any level, parties are notified. A practical example includes notices and alerts of vaccine delivery at the site and who delivered when vaccine supply reaches the tampered unit.

This eventing layer can use the above data layer to identify parameters from data attributes, like when and whom to send notifications. Such type of eventing layer already exists in modern web apps nowadays, specifically in machine learning.

\section{AI}
AI is a powerful technology with human-like intelligence that works as a catalyst in enhancing performance and efficiency. The application of AI in the healthcare sector is notably visible such as drug discovery. Scientists are rapidly implementing inexpensive AI-enabled technologies to find therapies. If given enough data, the machine can aid the search of the drug\cite{keshavarzi2020artificial}. Once we build the information that allows us to have quality data with few errors, we can use sophisticated math models and train machines using reinforcement learning. Hence, trained machines can help in predictions for forecasting inventory, preparing for hazards, analysing the turnaround time based on the optimal functioning of the overall supply chain. We describe these models here that can be fed in any application to achieve optimal distribution.

\begin{enumerate}
	\item Present vaccine distribution largely focuses on the individual at risk. We can use spatiotemporal regions for the newest cases of infection during a certain time frame and compare them with the standard practice of demographical vaccines distribution. A computational model presented in\cite{grauer2020strategic}, for a locally well-mixed population, strongly reduces the number of deaths by 35\%. The model emphasises the density of the population in a given space and time ie a region, targets using a math-based model and realises herd immunity.
	\item Typically, vaccines are distributed via a four-tier hierarchical network, the classic WHO-EPI model inherently is limited to meeting the demand and fixing the location of the clinic. Mixed-integer programming solves key issues of selecting storage sizes and ensures all delivery is made in a single trip\cite{yang2020optimizing}. This model is based on the following hubs and nodes associated with parental hubs. The below table shows the numerical output using the MIP model.
  
	\includegraphics[scale=0.4]{maths.png}
	\item The real-world experimentation of vaccine distribution is not possible as unknown problems can occur such as scarcity of vaccine or sub-components. With enough data supplied including contextual policies with demographics for fair distribution, the resulting dynamics may change over time. The VacSim model emphasises challenges that need to be solved in real-time, without ground truth availability. This model employs the classic reinforcement learning with the forward feed that can be optimised in real-time.\cite{awasthi2020vacsim}

\includegraphics[width=\textwidth]{model.png}
	\item 

\end{enumerate} 
 

\section{Cold chain consideration}

Cold chain is the storage method of the vaccine in a provided minimum temperature range. The vaccine is a biological product that becomes ineffective over a period of time as well as when exposed to a temperature higher/ lower than the threshold or in the open air, may lose its potency. There are various other reasons for vaccine waste, like in Delhi fewer people showing up than expected\footnote{https://www.hindustantimes.com/india-news/hesitancy-causes-wastage-of-1-000-vaccine-doses-in-delhi-say-health-officials-101611091051711.html} or the delivery site having only limited storage modes that do not last for long.

\subsection{Studies}

14 and 35\% of refrigerators or transport shipments were found to have exposed vaccines to freezing temperatures, while in studies examining all the distribution segments found between 75 and 100\% of the vaccine shipments were exposed.\cite{kartoglu2014tools}. Freezing occurs when vaccine in transit uses ice packs for storage, another alternative is to use cold water packs. When developers and manufacturers design equipment to store vaccines at locations throughout the supply chain and during transport, there might be a tendency to focus on the individual user rather than the entire supply chain. But, the storage equipment sits within a larger ecosystem and characteristics such as power requirements, internal capacity, and price can have reverberating effects throughout the supply chain. Vaccine supply chain performance and efficiency depend on the ability of the system to meet storage device needs (such as maintenance technicians and spare parts) in the field. For example, the value of a passive vaccine storage device, i.e. ones that do not require a power source, depends on how well the ice supply chain can be coordinated, device mobility, and how empty devices are swapped with refilled devices.\cite{lee2017importance}
 
Currently available vaccines possess shortcomings, such as the inefficient triggering of a cell-mediated immune response and the lack of protective mucosal immunity. In this regard, recent work has been focused on vaccine delivery systems, as an alternative to injectable vaccines, to increase antigen stability and improve overall immunogenicity. In particular, novel strategies based on edible or intradermal vaccine formulations have been demonstrated to trigger both a systemic and mucosal immune response. These novel vaccination delivery systems offer several advantages over injectable preparations, including self-administration, reduced cost, stability, and elimination of a cold chain \cite{criscuolo2019alternative}
 
 \subsection{Alternative Transformation}
 Nigeria was suffering from the lowest vaccination rates in the world due to the end-to-end approach. The nation used a strategy combining innovations in how data was captured, recorded and used to drive decision making and perform higher levels of uptime.\cite{sarley2017transforming}
 
 Methods like the HERMES computational simulation model for on-ground implementation like in the case of Mozambique resulted in 27\% vaccine delivery. The alternative system also produced higher availability at lower costs after new vaccine introductions.\cite{LEE20164998}
 
 In some situations, it is more efficient to bring the vaccines to the people instead of the other way around. Mobile medical teams need to route vaccines from one location to another. Halper and Raghavan define the mobile facility routing problem, with moving facilities to serve demand at different nodes in the network. In the case of multiple facilities, the routing problem is NP-hard and a heuristic is proposed to solve the problem.\cite{duijzer2018literature}
 
 To improve instant action and fallback strategies like what to do when a temperature falls below a certain level, constant data loggers as described above in the eventing layer section and data layer for synchronised behaviour across all domains are used.
 
 \subsection{Handling Waste}
 
Design of the medical waste supply chain is critical for supplies that have been used for infectious patients, where the risk of further transmission is always imminent. Pishvaee et al. (2014) proposed a method to design a sustainable medical supply chain, considering the complete life cycle of medical supplies and waste. Saif and Elhedhli (2016) took environmental considerations into account when studying the design of a cold supply chain. \cite{duijzer2018literature} The integration of environmental and social considerations should be incorporated while designing a waste management system.
 
 \subsection{Integration}
 The WHO recommends integrating the vaccine supply chain with other health supply chains. Yadav et al. (2014) studied the possibilities of such integration. According to their opinion, though outsourcing can be beneficial, it is important to consult all stakeholders in advance. The study provides illustrations of successful integration from which lessons can be learnt. The authors conclude that it is highly important to carefully determine which parts of the supply chain should be outsourced and to whom.\cite{duijzer2018literature}
 In India, there are various methods like using public and private transport infrastructure for delivery methods. Private companies like Uber and Ola have built technology to map faster routes with maximum optimisation of end to end delivery. One use case is utilising the existing fleet to transport vaccines internally in a local region.
\section{Cryptographic Methods}

Data privacy is one of the key parameters to be sure of while delivering vaccines throughout the network A process defines how data is governed internally and how it is going to be shared with stakeholders. This data includes health records and constant monitoring of the users, therefore data security is an important pillar in the vaccine network. Government systems are highly centralised in nature that trade-off the security requirements. 
	
\section{Challenges}

Recent studies indicate 71\% of individuals from 19 countries would take vaccines. Whereas, countries like Russia have only 55\% of people that are willing to participate in the exercise. Governments will have a tough time ahead convincing people to get vaccinated. Today, governments need to fundamentally gain trust and shift their view and directly motivate citizens with the use of effective data-driven planning and dialogue. Presently, countries like India have scientists from all across different domains who are raising voices against vaccine production and its efficacy since some vaccines did not even go to phase 3. 

\subsection{Transparency}

Vaccine builders and governments have to be open and clear about the data that it provides to users.

\subsection{Building for Future}

The government needs to start building platforms and systems with the future approach. Technology can be used where value driven systems are in place and citizens can access true information from the government about their health.	
  


\bibliographystyle{aabbrv}
\bibliography{bib/graph}


\end{document}




